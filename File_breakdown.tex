\documentclass[12pt, letterpaper]{article}

\usepackage[utf8]{inputenc}
\usepackage{xcolor}
\usepackage[margin=1 in]{geometry}

\begin{document}

\centerline{\textbf{Downloadable Files to Include With Each OpenMDAO Video}} 

\begin{itemize}

%%%----------Video 1----------%%%

	\item \underline{Video 1: Introduction to Optimization}
		\begin{itemize}
			\item Code: None
			\item Slides: Lecture
		\end{itemize}

%%%----------Video 2----------%%%
	\item \underline{Video 2: Details of Optimization}
		\begin{itemize}
			\item Code: None
			\item Slides: Lecture
		\end{itemize}

%%%----------Video 3----------%%%
	\item \underline{Video 3: Introduction to OpenMDAO} 
		\begin{itemize}
			\item Code: None
			\item Slides: Lecture
		\end{itemize}

%%%----------Video 4----------%%%
	\item \underline{Video 4: Details of OpenMDAO}
		\begin{itemize}
			\item Code: None
			\item Slides: Lecture (including installation instructions), assignment slide with OpenMDAO git repo url
		\end{itemize}

%%%----------Video 5----------%%%
	\item \underline{Video 5: Running with OpenMDAO}
		\begin{itemize}
			\item Code: Pre-coded paraboloid and quadratic models
			\item Slides: Lecture, assignment slide showing quadratic equation and the variables to print out
		\end{itemize}

%%%----------Video 6----------%%%
	\item \underline{Video 6: Explicit Components in OpenMDAO}
		\begin{itemize}
			\item Code: Lift component that was written in the video, problem statement for actuator component assignment
			\item Slides: Lecture,assignment slide with equations for actuator problem from OpenMDAO docs
		\end{itemize}

%%%----------Video 7----------%%%
	\item \underline{Video 7: Implicit Components in OpenMDAO} 
		\begin{itemize}
			\item Code: Quadratic component that was written in the video, problem statement for Kepler's equation assignment
			\item Slides: Lecture, assignment slide with Kepler's equation
		\end{itemize}

%%%----------Video 8----------%%%
	\item \underline{Video 8: Groups in OpenMDAO} 
		\begin{itemize}
			\item Code: Group from Lab 1 of 1-day OpenMDAO course, pre-written components and problem statement for Sellar problem
			\item Slides: Lecture, assignment slides with Sellar equations and a model hierarchy diagram
		\end{itemize}
		
%%%----------Video 9----------%%%
	\item \underline{Video 9: Problems in OpenMDAO and a Review of Model Hierarchy} 
		\begin{itemize}
			\item Code: Problem statement for Lab 1 of 1-day OpenMDAO course, 1 component of the Hohmann transfer example in the docs
			\item Slides: Lecture, assignment slide with the equation(s) for the Hohmann component
		\end{itemize}

%%%----------Video 10----------%%%
	\item \underline{Video 10: Putting it all together} 
		\begin{itemize}
			\item Code: Entire aircraft in flight equilibrium model that was written in the video
			\item Slides: Lecture
		\end{itemize}
		
%%%----------Video 11----------%%%
	\item \underline{Video 11: Connection vs. Promotion} 
		\begin{itemize}
			\item Code: Group from lab 1 of 1-day OpenMDAO course, group for Sellar problem using connection
			\item Slides: Lecture, assignment slide with Sellar equations
		\end{itemize}
		
%%%----------Video 12----------%%%
	\item \underline{Video 12: ExecComp and BalanceComp} 
		\begin{itemize}
			\item Code: Modified lift example and quadratic example from video, original actuator disk and Kepler's equation components and problem statements
			\item Slides: Lecture, assignment slide with the equations for the actuator disk and Kepler problems
		\end{itemize}

%%%----------Video 13----------%%%
	\item \underline{Video 13: Derivatives in OpenMDAO} 
		\begin{itemize}
			\item Code: Modified flight equilibrium code from Video 10 that includes derivatives, original actuator disk and Kepler's equation components and problem statements
			\item Slides: Lecture, assignment slide with the equations for the actuator disk and Kepler problems
		\end{itemize}

%%%----------Video 14----------%%%
	\item \underline{Video 14: Check\_partials} 
		\begin{itemize}
			\item Code: Components and problem statement for the paraboloid example with derivatives and check\_partials employed, heating component and problem statement from the shuttle reentry example in dymos with derivatives and check\_partials employed using cs, aerodynamics component from the shuttle reentry dymos example with some bad derivatives in it
			\item Slides: Lecture, assignment slide with equation(s) for the shuttle reentry component
		\end{itemize}

%%%----------Video 15----------%%%
	\item \underline{Video 15: Adding Vectors in OpenMDAO} 
		\begin{itemize}
			\item Code: Completed Sellar model with vectorized z-variables
			\item Slides: Lecture, assignment slide with Sellar model equations
		\end{itemize}
		
%%%----------Video 16----------%%%
	\item \underline{Video 16: Visualizing an OpenMDAO model} 
		\begin{itemize}
			\item Code: Completed code of aircraft in flight equilibrium from Video 13, completed Sellar model from the solution to the assignment in Video 8
			\item Slides: Lecture (including a slide with the syntax on how to call viewers), assignment slide with the equations for the Sellar model
		\end{itemize}

%%%----------Video 17----------%%%
	\item \underline{Video 17: The initialize method} 
		\begin{itemize}
			\item Code: Full model of Lab 1 from 1-day OpenMDAO training course, full model of aircraft in flight equilibrium with derivatives
			\item Slides: Lecture, assignment slide with the equations of the aircraft in flight equilibrium model
		\end{itemize}
		
%%%----------Video 18----------%%%
	\item \underline{Video 18: Revisiting the Structure of an OpenMDAO Model} 
		\begin{itemize}
			\item Code: None
			\item Slides: Lecture (including a slide with the equations for the Hohmann transfer example), assignment slide with the equations from the connection example in the 1-day OpenMDAO training course
		\end{itemize}	
		
%%%----------Video 19----------%%%
	\item \underline{Video 19: Building and Debugging a Model} 
		\begin{itemize}
			\item Code: Full but slightly broken model of the connection example in the 1-day OpenMDAO training course (include a singular matrix error, a couple bad derivatives, and a missed connection)
			\item Slides: Lecture, assignment slides including one with the equations from the connection example and one with the correct output from the connection example
		\end{itemize}
		
%%%----------Video 20----------%%%
	\item \underline{Video 20: Putting it all together (again)} 
		\begin{itemize}
			\item Code: Full model of lab 2 from the 1-day OpenMDAO training course
			\item Slides: Lecture, assignment slide with the equations for the nonlinear circuit analysis example
		\end{itemize}
		
%%%----------Video 21----------%%%
	\item \underline{Video 21: Validation} 
		\begin{itemize}
			\item Code: Full model with tests of the aircraft in flight equilibrium example, full code of the nonlinear circuit analysis model 
			\item Slides: Lecture, assignment slide with the equations for the nonlinear circuit analysis problem
		\end{itemize}
	
%%%----------Video 22----------%%%
	\item \underline{Video 22: Solvers in OpenMDAO} 
		\begin{itemize}
			\item Code: None
			\item Slides: Lecture (including link to the docs)
		\end{itemize}
		
%%%----------Video 23----------%%%
	\item \underline{Video 23: Drivers in OpenMDAO}
		\begin{itemize}
			\item Code: None
			\item Slides: Lecture (including link to the docs)
		\end{itemize}

%%%----------Video 24----------%%%
	\item \underline{Video 24: Case Recording OpenMDAO} 
		\begin{itemize}
			\item Code: Aircraft in flight equilibrium model with case recorders, nonlinear circuit analysis model
			\item Slides: Lecture, equations for nonlinear circuit analysis model
		\end{itemize}
		
%%%----------Video 25----------%%%
	\item \underline{Video 25: Scaling in MDO and OpenMDAO} 
		\begin{itemize}
			\item Code: Scaled model of displacement and stress equations, unscaled model of angular displacement and shear stress equations
			\item Slides: Lecture, assignment slide with equations for angular displacement and shear stress
		\end{itemize}
		
%%%----------Video 26----------%%%
	\item \underline{Video 26: Deeper Dive into Implicit Components} 
		\begin{itemize}
			\item Code: Sellar model with solve\_nonlinear written in, full model of explicit lift component from Video 6
			\item Slides: Lecture, assignment slide with equations for lift component
		\end{itemize}

%%%----------Video 27----------%%%
	\item \underline{Video 27: Advanced topics and further exploration into OpenMDAO} 
		\begin{itemize}
			\item Code: Lift example from 1-day OpenMDAO training course and quadratic example, after they have been modified in the video
			\item Slides: Lecture
		\end{itemize}
	
\end{itemize}

\end{document}
		