\documentclass[12pt, letterpaper]{article}

\usepackage[utf8]{inputenc}
\usepackage{xcolor}
\usepackage[margin=1 in]{geometry}

\begin{document}

\centerline{\textbf{OpenMDAO Lesson Plan}} 

\noindent \textit{\centerline{(Words in italics and parentheses are notes to the teacher.)}}
\newline
\centerline{\textcolor{blue}{Words in blue code problems that need to be included.}}
\newline \newline
\textit{(Note: make all the slides downloadable and available to students. Also make the completed coding examples from the videos downloadable to students. Aiming for about 50 minutes per video. It is important to note that the students may or may not be taking notes while watching these videos, so all pertinent information needs to be included in either the downloadaable code files or the downloadable slides (including all command line code).} \textit{Whenever code is shown, unless otherwise specified it is intended to be typed up and explained along the way.)} 

\begin{itemize}

%%%----------Video 1----------%%%

	\item \underline{Video 1: Introduction to Optimization} \textit{(This video is meant to get everyone up to speed on practical optimization issues, and doesn't have any actual coding in it, just slides).} 
		\begin{itemize}
			\item MDO
			\item Explain what MDO is and why it is useful.
			\item Explain what an objective is.
			\item Explain what a constraint is.
			\item Explain what a design variable is.
			\item Explain how to choose initial guesses for the optimizer and solver.
			\item Explain the differences between gradient-free and gradient-based optimization.
			\item \textbf{Assignment: None.}
		\end{itemize}

%%%----------Video 2----------%%%
	\item \underline{Video 2: Details of Optimization} \textit{(This video is meant to finish getting everyone up to speed on practical optimization issues, and still doesn't have any actual coding in it.)} 
		\begin{itemize}
			\item Explain what failed optimization cases are.
			\item Explain why failed cases happen and how to handle them.
			\item Discuss postprocessing of optimized data.
			\item Explain how and when to use equality constraints.
			\item \textbf{Assignment: None.}
		\end{itemize}

%%%----------Video 3----------%%%
	\item \underline{Video 3: Introduction to OpenMDAO} \textit{(This video is meant to provide a general introduction to what OpenMDAO is and does, and to start getting the student thinking about the modular and hierarchical structure of OpenMDAO. So again, no actual coding, just slides.)} \textit{(There are pre-existing slides on much of this video's content in the 1-day training course.)}
		\begin{itemize}
			\item Motivation behind using OpenMDAO.
			\item What OpenMDAO is useful for and can handle.
			\item How OpenMDAO thinks (hierarchical modularized models). Use examples. \textit{(Show lots of pretty pictures with lots of explanation to help them grasp the overall structure, this is very important.)}
			\item OpenMDAO has both solvers and drivers. Briefly explain what these do. They will be covered in more detail later in the course.
			\item Show a sample file \textcolor{blue}{(Lab 2 from 1-day training)} and explain the general layout of an OpenMDAO file. (ie. component, group, problem. Not super detailed.)
			\item \textbf{Assignment: None.}
		\end{itemize}

%%%----------Video 4----------%%%
	\item \underline{Video 4: Details of OpenMDAO} \textit{(This video is meant to clear up some of the more detailed introductory issues to OpenMDAO. The only actual coding is the demonstration of installation.)}
		\begin{itemize}
			\item Explain why OpenMDAO needs derivatives.
			\item Explain how OpenMDAO handles units.
			\item Show how to install OpenMDAO both ways, explain which is better in certain circumstances. \textit{(May not be necessary to type this out. Probably acceptable to have a couple slides with the syntax and walk through them.)}
			\item \textbf{Assignment: Install OpenMDAO.}
		\end{itemize}

%%%----------Video 5----------%%%
	\item \underline{Video 5: Running with OpenMDAO} \textit{(For this video, have some slides but also open a pre-coded file of the \textcolor{blue}{paraboloid} example. No need to code up the importing portion, they likely will already be familiar with how that works, or it is straightforward enough. Do actually run the problem and talk over the output it generated. Also actually code up the accessing and printing values portion, explaining what each line does as you write it.)}
		\begin{itemize}
			\item Importing OpenMDAO classes. \textit{(Have a slide about how this is done, and then open up the example code and show where it is written/how it looks.)}
			\item Importing components from pre-existing files \textit{(Have a slide about how this is done, and then open up the example code and show where it is written/how it looks.)}
			\item Running a sample model. \textit{(Actually do this so they can see what it looks like and how to run a model.)}
			\item Understanding the sample model output. Walk through the output generated by the code that was run.
			\item Accessing and printing out values from the model. Show how this is done by isolating each variable in the model one by one and printing it out. \textit{(Talk through it as the code is written.)}
			\item \textbf{Assignment: Run the \textcolor{blue}{quadratic} model and print out each out the variables.} \textit{(Need to provide code for quadratic model, along with a slide showing each of the variables in the model so they know what to print out.)}
		\end{itemize}

%%%----------Video 6----------%%%
	\item \underline{Video 6: Explicit Components in OpenMDAO} \textit{(For this video, have some slides at the beginning illustrating what the 				component does and what the different methods do. Then start with a blank page and type code out on the screen while illustrating exactly what each line does. The \textcolor{blue}{Lift computation example the 1-day training} should work well for this video.)}
		\begin{itemize}
			\item Explicit components are components where the outputs are defined as explicit functions of the inputs.
			\item Two necessary methods: setup and compute. \textit{(Show on slide what each method does and explain it a little.)}
			\item Setup is where the inputs and outputs are declared, along with their derivatives.
				\begin{itemize}
					\item Explain how to choose which variables are inputs and outputs.
					\item Show how add inputs and outputs. \textit{(Type it out and explain as you go.)}
					\item Show how to add defaults, descriptions, and units to the inputs and outputs. \textit{(Type it out and explain as you go.)}
					\item Explain how to determine what outputs need derivatives with respect to what inputs.
					\item Show how to declare derivatives, including derivative type. For now we will finite difference all our derivatives. \textit{(Type it out and explain as you go.)}
				\end{itemize}
			\item Compute is where the actual explicit equation is defined.
				\begin{itemize}
					\item Show how to make the method. \textit{(Type it out and explain as you go.)}
					\item Show what parameters to pass. \textit{(Type it out and explain as you go.)}
					\item Type up the equation. \textit{(Type it out and explain as you go.)}
					\item Show what you would do if you had more than one output (ie. more than one equation to type).
				\end{itemize}
			\item \textit{(Run the model so they can see the output. In order to do this there will need to be a problem defined, so just copy paste the problem code onto the screen and quickly explain that it will be demonstrated later how that code was generated.)}
			\item \textbf{Assignment: Build a simple explicit component using the \textcolor{blue}{provided actuator disk equations}. It is only necessary to use the setup and compute methods, just finite difference the derivatives.} \textit{(Need to provide partial code file with the problem instantiation already written up, so all they have to do is build the component and they can run it.)}
		\end{itemize}

%%%----------Video 7----------%%%
	\item \underline{Video 7: Implicit Components in OpenMDAO} \textit{(For this video, have some slides at the beginning illustrating what the component does and what the different methods do. Then type code out on the screen while illustrating exactly what each line does. The 					\textcolor{blue}{quadratic example from the docs} should work well for this video.)}
		\begin{itemize}
			\item Implicit components are components where the outputs are not explicit functions of the inputs.
			\item These components need to be used if the desired outputs cannot be written as explicit functions of the inputs.
			\item Two necessary methods: setup and apply\_nonlinear. \textit{(Show on slide what each method does and explain it a little.)}
			\item Setup is where the inputs and outputs are declared, along with their derivatives.
				\begin{itemize}
					\item Explain how to determine which variables are inputs and which are outputs.
					\item Show how to add inputs and outputs. \textit{(Type it out and explain as you go.)}
					\item Show how to add defaults, descriptions, and units. \textit{(Type it out and explain as you go.)}
					\item Explain how to determine what outputs need derivatives with respect to what inputs.
					\item Show how to declare derivatives, including derivative type. For now we will finite difference all our derivatives. \textit{(Type it out and explain as you go.)}
				\end{itemize}
			\item Apply\_nonlinear is where the implicit equation(s) is(are) defined. 
			\item Instead of writing explicit equations as was done in the explicit component, here each equation is defined by a residual. There is a residual for each output, but even though the name of the residual follows the name of the output it corresponds to, the actual value is will be zero, and the output need not appear in its own residual.
			\item \textit{(Run the model so they can see the output. In order to do this there will need to be a problem defined, so just copy paste the problem code onto the screen and quickly explain that it will be demonstrated later how that code was generated.)}
			\item \textbf{Assignment: Build a simple implicit component \textcolor{blue}{using the transcendental Kepler's equation from the docs}. It is only necessary to use the setup and apply\_nonlinear methods, just finite difference the derivatives.} \textit{(Need to provide partial code file with the problem instantiation already written up, so all they have to do is build the component and they can run it.)}
		\end{itemize}

%%%----------Video 8----------%%%
	\item \underline{Video 8: Groups in OpenMDAO} \textit{(For this video, have some slides at the beginning illustrating what the group does and what the different methods do. Then type code out on the screen while illustrating exactly what each line does. The \textcolor{blue}{Lab 1 group from the 1-day training} should work well for this video. Don’t go through and define the components for the problem, just present a slide with the equations and explain that the components have already been written according to the methods previously discussed, then focus on using the group to tie them together.)}
		\begin{itemize}
			\item Show how groups fit into the OpenMDAO structure. \textit{(Use a slide to illustrate this.)}
			\item The basic group operations happen in the setup method.
			\item Show how to add an IndepVarComp and explain what it does. \textit{(Type it out and explain as you go.)}
			\item Show how to add components. \textit{(Type it out and explain as you go.)}
			\item Show how to add a subgroup. \textit{(Type it out and explain as you go.)}
			\item Show how to connect inputs and outputs via connection. \textit{(Type it out and explain as you go.)}
			\item Show how to add a solver to a subgroup, and explain why one is needed here. \textit{(Type it out and explain as you go.)}
			\item Show how to add an objective and constraint. \textit{(Type it out and explain as you go.)}
			\item \textit{(Run the model so they can see the output. In order to do this there will need to be a problem defined, so just copy paste the problem code onto the screen and quickly explain that it will be demonstrated later how that code was generated.)}
			\item \textbf{Assignment: The provided equations have already been written up into components. Write a group for the components.} \textit{(Need to provide components as well as problem for the group, so they can run the model after writing the group. Also provide a slide with the equations that the components utilize, as well as a model hierarchy diagram that illustrates how the provided code fits together.)} \textcolor{blue}{Use the sellar problem.}
		\end{itemize}
		
%%%----------Video 9----------%%%
	\item \underline{Video 9: Problems in OpenMDAO and a Review of Model Hierarchy} \textit{(For this video start by saying that today we are going to learn to bring everything together. Then \textcolor{blue}{use the Lab 1 group from the 1-day training} to write a problem for. After that code has been run, use slides to show how each of the models we have developed thus far fits together with its problem, group (if necessary) and solver/driver.)}
		\begin{itemize}
			\item The problem is the container for the entire OpenMDAO model.
			\item Show how to instantiate a problem. \textit{(Type it out and explain as you go.)}
			\item Show how to add a component/group. \textit{(Type it out and explain as you go.)}
			\item Show how to set up a solver. \textit{(Type it out and explain as you go.)}
			\item Show how to set up a driver. \textit{(Type it out and explain as you go.)}
			\item Show how to add design variables. \textit{(Type it out and explain as you go.)}
			\item Show how to add constraints. \textit{(Type it out and explain as you go.)}
			\item Show how to add objectives. \textit{(Type it out and explain as you go.)}
			\item \textit{(Run the problem so they can see the output.)}
			\item Show the slide with the general OpeMDAO hierarchy again and remind them of how the overall structure works. Then put up a slide for each of the previous lecture examples showing the hierarchy of how that example would look in a problem. Include any solvers/drivers. Talk through each slide and explain why it is broken up the way it is.
			\item \textbf{Assignment: Download the provided component and write a problem for it, then run it. For extra practice, try writing your own problems for the two components and one group that you made in previous assignments!} \textcolor{blue}{Use one of the components from the Hohmann transfer example.}
		\end{itemize}

%%%----------Video 10----------%%%
	\item \underline{Video 10: Putting it all together} \textit{(This video doesn’t teach any new content, it is an extended example on how to use what has been taught so far to put together a simple model. For this example it would be straightforward to use the \textcolor{blue}{two equations of the aircraft in flight equilibrium}. Just make sure to write one explicit component and one implicit component.)}
		\begin{itemize}
			\item Show (on a slide) the equations chosen, and how they will be used (ie. what are the inputs and outputs of each component, and how are the components broken up?)
			\item Show (on a slide) a picture of the model structure for the model to be built, including any solvers/drivers.
			\item Show how to make each of the two components, making sure to only use the setup and compute/apply\_nonlinear methods, and use fd derivatives. \textit{(Type it out and explain as you go.)}
			\item Show how to make the group to put the two components in, and put a solver in it. \textit{(Type it out and explain as you go.)}
			\item Show how to make the problem. \textit{(Type it out and explain as you go.)}
			\item \textit{(Run the model so they can see the output. Make a bit of a fuss over the fact that they now know how to build their own simple model. Get them excited about realizing how much they have learned.)}
			\item \textbf{Assignment: None.} 
		\end{itemize}
		
%%%----------Video 11----------%%%
	\item \underline{Video 11: Connection vs. Promotion} \textit{(For this video, slides are used to illustrate the difference between connection and promotion, and then code is used to demonstrate how to implement promotion. The group from \textcolor{blue}{Video 8} should work well here.)}
		\begin{itemize}
			\item Another way to write groups is to connect variables via promotion instead of connection.
			\item Connection explicitly ties two variables from different components together, but promotion instead raises the level of the variable name from the components up to the group.
			\item Promoted variables with the same name are automatically connected.
			\item Show how to rewrite the group from Video 8 using promotion instead of connection.
			\item Run the model so they can see that the output hasn't changed.
			\item \textbf{Assignment: Rewrite the group from the \textcolor{blue}{Video 8} assignment using promotion instead of connection.} \textit{(Need to provide solution to Video 8 assignment as a dowloadable code file in case the didn't do it or didn't do it correctly. Also provide a slide with the equations that the components utilize, as well as a model hierarchy diagram that illustrates how the provided code fits together.)}
		\end{itemize}
		
%%%----------Video 12----------%%%
	\item \underline{Video 12: ExecComp and BalanceComp} \textit{(For this video, slides are used to introduce the concept of each component and provide a little information. Then, type the code out on the screen while illustrating what each line does in order to provide an example implementation for each component. For the ExecComp modify the code written in \textcolor{blue}{Video 6}, and for the BalanceComp modify the code written in \textcolor{blue}{Video 7})} 
		\begin{itemize}
			\item ExecComp is a shortcut to an explicit component. 
			\item It doesn’t require all the setup of a regular explicit component, it just needs an equation.
			\item It assumes outputs on the left and inputs on the right. 
			\item It automatically uses fd for the derivatives. 
			\item Show a slide with the paraboloid equation to remind them of it.
			\item Show the code where the paraboloid component is called (from Video 4) and show how to modify it to use an ExecComp. \textit{(Type out the modification and explain as you go.)}
			\item \textit{(Run the model so they can see the output.)}
			\item Similarly, BalanceComp is a shortcut to an implicit component.
			\item It doesn’t require all the setup of a regular implicit component.
			\item Show a slide with the quadratic equation to remind them of it.
			\item Show the code where the quadratic component is called (from Video 7) and show how to modify it to use a BalanceComp. \textit{(Type out the modification and explain as you go.)}
			\item \textit{(Run the model so they can see the output.)}
			\item \textbf{Assignment: Rewrite the explicit component from the \textcolor{blue}{Video 6 assignment} using an ExecComp. Rewrite the implicit component from the \textcolor{blue}{Video 7} assignment using a BalanceComp. Run each one and compare your answers to what you got in Videos 6 and 7.} \textit{(Provide downloadable code files of the solutions to the assignments in Videos 6 and 7 in case they either didn't do them or didn't do them correctly.)}
		\end{itemize}

%%%----------Video 13----------%%%
	\item \underline{Video 13: Derivatives in OpenMDAO} \textit{(For the coding in this video modify the code written in \textcolor{blue}{Video 10}.)}
		\begin{itemize}
			\item Have slides and discuss the benefits and drawbacks of fd, cs, and analytic derivatives in OpenMDAO. \textit{(There should already be slides on this in “getting\_derivatives\_in\_OpenMDAO.tex” in the OpenMDAO training materials repo, and duplicates in John Jasa’s “getting\_derivatives\_presentation.pptx”.)}
			\item Derivatives are implemented similarly but slightly differently in explicit and implicit components. 
			\item In order to implement them as cs instead of fd, the only thing that needs changing from before is that the method declared is cs instead of fd. Implementing analytic derivatives is a bit more involved.
			\item How to declare derivatives was covered in previous videos, but now instead of being declared as fd they will be declared as analytic (the default if no method is specified).
			\item For explicit components, the compute\_partials method is used to define analytic derivatives. 
			\item Show a slide with the partial derivatives of the Sellar problem.
			\item Show how to implement the compute\_partials method for the explicit component. \textit{(Type and explain what you are doing as you go. Run the model so they can see if the output has changed. Also change the method back to fd as well as to cs, and run the model in all cases so they can observe whether or not the output changes. Afterwards return the method to fd so that the impact of analytic derivatives on the implicit component alone can be explored next.)}
			\item Derivatives in implicit components are slightly different, because instead of taking the derivative of outputs with respect to inputs, we take the derivative of residuals with respect to both outputs and inputs.
			\item The mechanics of declaring and providing the analytic derivatives are the same in explicit and implicit components, but in implicit components the derivatives are provided in the linearize method.
			\item Show how to implement the linearize method for the implicit component. \textit{(Type and explain what you are doing as you go.)}
			\item \textit{(Run the model so they can see if the output has changed. Also change the method back to fd as well as to cs, and run the model in all cases so they can observe whether or not the output changes. Then go back and turn on the analytic derivatives in the Explicit Component as well as here in the implicit component, run the model, and show them if the output changes from what it originally was.)}
			\item \textbf{Assignment: Implement analytic derivatives in the models you made in \textcolor{blue}{Videos 6 and 7}, and run them. Do the results change? Why or why not?} \textit{(Need to provide downloadable code for the assignments from Videos 6 and 7 in case they either didn't do them or didn't do them correctly.)}
		\end{itemize}

%%%----------Video 14----------%%%
	\item \underline{Video 14: Check\_partials} \textit{(This video is mostly coding, there will be a few slides at the start to give a bit of information, but then it is mostly coding demonstrations of how to use check\_partials in different ways.)}
		\begin{itemize}
			\item Introduce check\_partials and explain what it does.
			\item Show how to use check\_partials by coding it into \textcolor{blue}{the paraboloid example}. \textit{(Type it out and explain as you go.)} 
			\item Check\_partials has several settings to help you. One very useful setting is to tell it to use cs derivatives instead of fd (which is the default).
			\item Show how to tell check\_partials to use cs derivatives. Code an example of derivatives exceeding their tolerance with fd derivatives but meeting it with cs. \textcolor{blue}{For this example use the heating component from the dymos shuttle reentry model.} \textit{(Type it out and explain as you go.)} 
			\item \textbf{Assignment: Download the provided code \textcolor{blue}{which is the aerodynamics component from the dymos shuttle reentry model}. Some of the derivatives are bad. Run check\_partials to find out which ones they are, and fix them. Use check\_partials to make sure you fixed them correctly.} \textit{(Need to provide downloadable code with some bad partials.)}
		\end{itemize}

%%%----------Video 15----------%%%
	\item \underline{Video 15: Adding Vectors in OpenMDAO} \textit{(For this problem start out with a few slides talking about the first couple bullet points. For the coding examples, use the \textcolor{blue}{Sellar problem from the Video 8 assignment} and show how to modify it to vectorize everything.)}
		\begin{itemize}
			\item 	Sometimes it is necessary to have inputs and/or outputs that are in vector form. OpenMDAO was made to handle this, however, we can specify derivatives in a little different way to help OpenMDAO out with vectors.
			\item Explain what the rows and columns of a Jacobian represent. \textit{(Jasa's slides on getting derivatives in OpenMDAO have some visuals on this already.)}
			\item It is easier for OpenMDAO if we specify which values in the Jacobian are nonzero.
			\item Explain how to determine which values in the Jacobian are nonzero.
			\item Show a reminder slide with the equations and modular structure of the Sellar problem.
			\item Show how to write the Sellar code in the setup method to vectorize the z-variable. \textit{(Write the code with one explicit and one implicit component, so they can see how it works in both. Type it up and explain as you go.)}
			\item Show how to write the Sellar code in the compute and apply\_nonlinear methods to vectorize the z-variable.  \textit{(Type it up and explain as you go.)}
			\item Show how to write the Sellar code in the compute\_partials and linearize methods to vectorize the derivatives. \textit{(Type it up and explain as you go.)}
			\item \textit{(Run the model so they can see the output.)}
			\item \textbf{Assignment: Vectorize the y-variable in the \textcolor{blue}{Sellar code} and run the model to make sure that the output hasn’t changed.} \textit{(Will need to provide downloadable file of Sellar code, likely put in separate place for the assignment than where all the downloadable lecture code is.)}
		\end{itemize}
		
%%%----------Video 16----------%%%
	\item \underline{Video 16: Visualizing an OpenMDAO model} \textit{(For this video, have several slides showing the different visualization options and comparing them to a block diagram. For the coding use the \textcolor{blue}{aircraft in flight equilibrium example from Video 10}.)}
		\begin{itemize}
			\item We are now at the point where we are able to build more detailed models, and it is important to be able to visualize them.
			\item OpenMDAO has a few different tools to aid in model visualization.
			\item N2 diagrams are very useful in visualizing models and checking for missed connections.
			\item Show what an N2 diagram looks like and put it next to a model block diagram to show how they relate (the Sellar problem would make a good example).
			\item Show what an unconnected variable looks like and explain what it means.
			\item Show how the N2 can be used to determine what solvers are where.
			\item Show how to generate the N2 for the example problem. \textit{(Type the command out and explain it.)}
			\item The connection viewer can also be helpful in visualizing a model and catching mismatched units.
			\item Show what a connection viewer looks like and put it next to the N2 from before to show how they relate.
			\item Show how to generate the connection viewer for the example problem.
			\item \textbf{Assignment: Generate an N2 for the \textcolor{blue}{model from the Video 8 assignment}. Also generate a connection viewer for it. Do you understand what these are saying? Have you missed any connections?} \textit{(Need to have downloadable code solution to the Video 8 assignment in case they didn't do it or didn't do it correctly.)}
		\end{itemize}

%%%----------Video 17----------%%%
	\item \underline{Video 17: The initialize method} \textit{(For this video, start with a few slides to help illustrate the pupose of the initalize method. Then code along to show them how to use it. Use the \textcolor{blue}{Lab 1 code from Video 8} to demonstrate with.)}
		\begin{itemize}
			\item Often times there is a need to pass certain parameters other than variables (such as constants or vector sizes) to a component or group.
			\item This can be done using the initialize method, which is written the same in both components and groups.
			\item The initialize method is traditionally implemented at the top of a component or group.
			\item To demonstrate the implementation of the initialize method, we will use our previously developed \textcolor{blue}{Lab 1 code}, and modify it so that we can multiply the equation in the explicit component by a constant which is set at the problem level. 
			\item First, go into the explicit component and walk through adding the initialize method. Include a default value of 1, and explain what the default is and does. \textit{(Type it out and explain as you go.)}
			\item Next, go into the group that contains the explicit and implicit components and show how to set up the initialize method in the group, and pass that information through the creation of the explicit component. \textit{(Type it out and explain as you go.)}
			\item Finally, go to the code where the problem is defined and show how to set the desired constant upon instantiation of the group. Then run the code with a few different constants to show them whether or not it changes, and explain why. \textit{(Type it out and explain as you go.)}
			\item Explain how using what you just wrote, you can modify the constant at the problem level and it passes all the way through the group level and down to the component.
			\item \textbf{Assignment: Use the group and components from the \textcolor{blue}{aircraft in flight equilibrium} and add the initialize method to the group and all the components. Modify the code so that a constant can be passed via the initialize method to multiply one of the equations. Then modify the problem statement so that you can control this constant at the problem level. Run the code with a few different constants. Does the output change? Why or why not?} \textit{(Need to include a downloadable code.)}
		\end{itemize}
		
%%%----------Video 18----------%%%
	\item \underline{Video 18: Revisiting the Structure of an OpenMDAO Model} \textit{(This video is all about teaching them to take a real-world problem and make it into the form of an OpenMDAO model. No actual code will be written, everything will be done on slides.)}
		\begin{itemize}
			\item Show the slides from Video 2 on “How OpenMDAO thinks” and briefly remind them of the model structure.
			\item Now it is time to discuss how to take a real-world problem and organize it into an OpenMDAO model.
				\begin{itemize}
					\item Explain how to determine the objective.
					\item Explain how to determine which equations should be component equations.
					\item Explain how to determine which equations should be constraint equations.
				\end{itemize}
			\item Once the governing equations, constraints, and objective have been selected, it is time to determine how to break them all up into components and groups.
			\item Explain what a good rule of thumb is for the number of inputs and outputs a component should have, and how related the equations need to be in order to be placed in the same component.
			\item Explain how to determine when/where to draw the boundaries for a group, and how this works into determining where to put the solvers.
			\item Walk through an example of using the method just described. \textcolor{blue}{The Hohmann transfer example from the docs should work well for this.}
			\item Show a slide with the general architecture of how to organize a complex model. \textit{(Perhaps the organization of the aviary model would be a good example.)}
			\item \textbf{Assignment: \textcolor{blue}{Provide them with the equations from the connect example from the 1-day training course.} Take the given equations and determine what the objective, constraints, and component equations are. Then decide how you would break these components equations up into implicit components, explicit components, and groups. (Note: no need to actually code the model unless you want to, just determine how to break it up.)} \textit{(Need to provide the equations in a downloadable file.)}
		\end{itemize}	
		
%%%----------Video 19----------%%%
	\item \underline{Video 19: Building and Debugging a Model} \textit{(Have slides to support the points listed below. There is no  actual coding in this video.)}
		\begin{itemize}
			\item Building an OpenMDAO model, just like building any other sort of computer program, requires constant error checking along the way.
			\item Once the model structure and equation locations have been determined, it is easiest to start small and build up to your full model.
			\item A simple way to do this is to build one component at a time. After each component, put it in a problem and run it to make sure it works. Then once you know it works you can integrate it into its group, run that group, and make sure it works too. So on and so forth.
			\item In order to make finding errors easier, it often is helpful to fd your derivatives at first. Just make sure to come back and implement them analytically later if you want to use analytic derivatives.
			\item Once you start implementing analytic derivatives, make sure to run check\_partials along the way. A significant number of model problems come from bad derivatives.
			\item A common problem that arises when building models is singular matrices. If you get a singular matrix error it means that you have a row or column of zeros in your Jacobian.
			\item A row of zeros means that you have an output that is unaffected by any of the inputs.
			\item A column of zeros means that you have an input which has no effect on any of the outputs.
			\item \textcolor{blue}{Provide a slightly broken example of the Hohmann transfer from the docs}, and explain what’s wrong/how to fix it. Use slides, not code. Use visuals to show what the equations, inputs, and outputs are, and this should make it apparent where the problem is. However, make sure to point that out anyway.
			\item \textbf{Assignment: \textcolor{blue}{Provide the connection example from the 1-day training.} The provided example code has errors in it. Figure out what the errors are and fix them. The correct answer is \textit{(provide correct answer here).}} \textit{(Need to include downloadable code with the error-filled example for them to fix. There should be a singular matrix error and a couple bad derivatives. Perhaps there should be a missed connection too, but this might be a bit much.)}
		\end{itemize}
		
%%%----------Video 20----------%%%
	\item \underline{Video 20: Putting it all together (again)} \textit{(This video doesn’t teach any new content, it is an extended example on how to use what has been taught so far to put together a more detailed model. Make sure to make use of the initialize method and analytic derivatives, as well as promotion and if possible vectorization.)} \textit{(Note: this video is by nature very long, and perhaps should be broken up into two videos.)}
		\begin{itemize}
			\item \textcolor{blue}{Use Lab 2 from the 1-day training.} \textit{Make sure it uses one implicit and one explicit component, requires a constant to be passed via the initialize method, and vectorization if possible. Use analytic derivatives, and make sure to generate an N2.)}
			\item Show (on a slide) the equations chosen and their derivatives. Walk through breaking them up into a model. (ie. what are the constraints, how will the components be broken up, what are the inputs and outputs, what can be vectorized, and what needs to be passed via the initialize method.)
			\item Show (on a slide) a picture of the model structure for the model to be built.
			\item \textit{(As the model is being built, demonstrate the ``start small'' approach that was previously taught (ie. first fd everything, build one component then test it, etc.).)}
			\item Build each of the two components, making sure to only used the setup, compute/apply\_nonlinear, initialize, and compute\_partials/linearize methods. After each component, put it in a problem and test it for bugs. \textit{(Type it out and explain as you go.)}
			\item Build a group to put the two components in. \textit{(Type it out and explain as you go.)}
			\item Build a problem. \textit{(Type it out and explain as you go.)}
			\item Generate an N2 and show them how to check the connections. \textit{(Maybe even intentionally miss a connection so they can see the value of the N2.)} \textit{(Type out generating the N2.)}
			\item Implement the analytic derivatives and run check\_partials. \textit{(Maybe even intentionally miss a partial or two so they can see how valuable check\_partials is.)}
			\item \textit{(Run the model so they can see the output. Make a bit of a fuss over the fact that they can now build a more complex model using governing equations, and they can implement analytic derivatives and vectors, and can control the model more from the top level. Get them excited about realizing how much they have learned.)}
			\item \textbf{Assignment: Use the provided equations to implement your own model with analytic derivatives. Build the model and run it to see the output.} \textcolor{blue}{Give them the nonlinear circuit analysis problem from the docs.} \textit{(Will need to include a downloadable file with the governing equations of the problem for them to implement.)}
		\end{itemize}
		
%%%----------Video 21----------%%%
	\item \underline{Video 21: Validation} \textit{(For this video use the \textcolor{blue}{the aircraft in flight equilibrium example from Video 10.})} 
		\begin{itemize}
			\item Model validation is a useful tool to help make sure a model isn’t accidentally broken after it has been written.
			\item This can be achieved by writing tests.
			\item Tests can be written at different levels. There can be a test for each component as well as tests for groups. 
			\item There are two portions to a test. There is a portion written in the same file as the component or group, and a portion written in a different file. \textit{(Note: this is how I learned to write tests, but I know there are a few different ways and so this can be modified based on what you want taught.)}
			\item Show how to write the in-component portion of a test for one of the components. \textit{(Type it out and explain as you go.)}
			\item Show how to write the independent-file portion of a test for the same component. \textit{(Type it out and explain as you go.)}
			\item Copy paste both portions of a test in to the rest of the components (and their accompanying test files), and explain that they are written essentially the same as the first component.
			\item Show how to write both the in-component portion and the independent-file portion of a test for the group. \textit{(Type it out and explain as you go.)}
			\item Show how to run testflo .
			\item \textbf{Assignment: Write tests for each component and the group you developed in the assignment accompanying \textcolor{blue}{Video 20}.} \textit{(Need downloadable code file of the solution to the Video 20 assignment in case they either didn't do it or didn't do it correctly.)}
		\end{itemize}
	
%%%----------Video 22----------%%%
	\item \underline{Video 22: Solvers in OpenMDAO} \textit{(This video is purely slides, no coding.)}
		\begin{itemize}
			\item OpenMDAO has two general types of solvers, linear and nonlinear.
			\item Explain the difference between linear and nonlinear solvers, and when you need each one. 
			\item Give a top-level view of how the different solvers work and when they are or are not useful.
			\item Explain some common specific solver settings and options that might come in handy.
			\item Explain how to appropriately select where to put solvers in the model.
			\item Explain how BalanceComp uses solvers. 
			\item \textbf{Assignment: Check out the OpenMDAO docs and learn more about the details of different solvers.} \textit{(Include link to docs where solvers are.)}
		\end{itemize}
		
%%%----------Video 23----------%%%
	\item \underline{Video 23: Drivers in OpenMDAO} \textit{(This video is purely slides, no coding.)}
		\begin{itemize}
			\item OpenMDAO has both solvers and drivers, and they do different things.
			\item Explain the difference between solvers and drivers, and give an example or two of how they would be implemented to complement each other.
			\item Give a top-level view of how the different drivers work and when they are or are not useful.
			\item Explain some common specific driver settings and options that might come in handy.
			\item \textbf{Assignment: Check out the OpenMDAO docs and learn more about the details of different solvers.} \textit{(Include link to docs where drivers are.)}
		\end{itemize}

%%%----------Video 24----------%%%
	\item \underline{Video 24: Case Recording OpenMDAO} \textit{(This video will need to be accompanied by coding. The \textcolor{blue}{aircraft in flight equilibrium example from Video 10} should work well for this.)}
		\begin{itemize}
			\item Sometimes it is helpful to see what is going on with a solver or driver. This is where case recorders come in.
			\item Case recorders can be attached to both solvers and drivers.
			\item Explain what case recorders do. \textit{(There are some already developed slides from the one-day OpenMDAO course to help illustrate case recording.)}
			\item Show how to add a case recorder to the Sellar problem solver, and run it. \textit{(Type it out and explain as you go.)}
			\item Show the Sellar problem case recorder output, and explain what each part means.
			\item Show how to add a case recorder to the Sellar problem driver, and run it. Highlight the differences between adding the recorder to a solver and a driver. \textit{(Type it out and explain as you go.)}
			\item Show the Sellar problem case recorder output, and explain what each part means. Point out the differences and commonalities with solver case recorder output.
			\item \textbf{Assignment: Attach a case recorder to the solver on the model generated in \textcolor{blue}{Video 8}. Run the model with the case recorder and look at the output. What does the output mean? Then repeat with a case recorder on the driver.} \textit{(Need to include a downloadable code file of the solution to the Video 8 assignment in case they didn't do it or didn't do it correctly.)}
		\end{itemize}
		
%%%----------Video 25----------%%%
	\item \underline{Video 25: Scaling in MDO and OpenMDAO} \textit{(The goal of this video is to teach about the importance of scaling in optimization in general, as well as its implementation in OpenMDAO. Most of the video is slides illustrating how scaling works and its importance. Only near the end when the implementation of scaling in OpenMDAO is demonstrated is there live coding.)}
		\begin{itemize}
			\item Show examples (equations with mismatched magnitudes, etc.) of poorly scaled optimization problems.
			\item Explain what the problem of scaling is, and why it is significant. Explain what can happen to a poorly scaled problem.
			\item Explain how scaling can help make problems better suited to optimization.
			\item Show how scaling would be implemented on the poorly scaled example equations that were used at the beginning of the video.
			\item Explain some good rules of thumb for determining appropriate scale factor starting points.
			\item Show how scaling can be implemented in OpenMDAO by using a poorly scaled toy problem. \textcolor{blue}{Cantilever beam deflection and stress in the beam make good coupled equations that have drastly different magnitudes.} \textit{(Show the already written code for the problem and briefly walk them through each line to get them oriented to what they are looking at. Then run the problem and show the output it generates without scaling. Finally walk them through determining the scale factors and modifying the code to implement scaling, and run the problem again to show them how the output has changed.)}
			\item \textbf{Assignment: The code and equations provided represent a poorly scaled problem. \textcolor{blue}{Angular displacement and shear stress in a rod made good coupled equations that have drastly different magnitudes.} Run the problem and then implement what you think are good scale factors. Then run the problem again to see if the output has changed.} \textit{(Need to provide downloadable code with poorly scaled equations.)}
		\end{itemize}
		
%%%----------Video 26----------%%%
	\item \underline{Video 26: Deeper Dive into Implicit Components} \textit{(For this video, the only coding required is to demonstrate the solve\_nonlinear method. For this demonstration use the implicit component in the \textcolor{blue}{Sellar example from Video 15}. The rest of the video uses only slides and explanations.)}
		\begin{itemize}
			\item There are some circumstances where it is helpful or a good idea to use implicit components even if they are not necessary.
			\item Explain the circumstances where it is beneficial to write a component as implicit, even if it not strictly necessary. 
			\item Implicit components also have another method which can be helpful. This is the solve\_nonlinear method.
			\item The solve\_nonlinear method gives a way to explicitly define an output as a function of inputs within an implicit component.
			\item Sometimes it is not even possible to define solve\_nonlinear, depending on the problem. However, it can be useful.
			\item Give some examples of common times when it is useful to define the solve\_nonlinear method.
			\item Show how to implement the solve\_nonlinear method in the implicit component in the \textcolor{blue}{Sellar example}. \textit{(Type this up and also explain as you go. Do this by modifying the already written implicit component in the \textcolor{blue}{Sellar example from Video 15}.)}
			\item Show how to isolate and call the solve\_nonlinear method. \textit{(Type this up and explain as you go.)}
			\item \textbf{Assignment: Rewrite the explicit \textcolor{blue}{lift component from Video 6} as an implicit component. Implement the solve\_nonlinear method. Isolate and call this method.} \textit{(Need to include a downloadable code file with the solution to the Video 6 assignment since they may not have done it or may not have done it correctly.)}
		\end{itemize}

%%%----------Video 27----------%%%
	\item \underline{Video 27: Advanced topics and further exploration into OpenMDAO} \textit{(Most of this video is slides, however, it does require some coding. For the demonstration of different settings in ExecComp and BalanceComp use the \textcolor{blue}{lift example and quadratic codes from video 9.})} \textit{(This can be expanded or contracted as much as you want. Perhaps add some more topics to tell them about? Also need to go into more detail on what is already here).}
		\begin{itemize}
			\item Using surrogate models in OpenMDAO. \textit{(There are pre-existing slides on this, althought they may not go into as much detail as is needed.)}
			\item File wrappers. \textit{(There are also pre-existing slides on this, although they also may not go into as much detail as is needed.)}
			\item Show different types of settings and equations for ExecComp and BalanceComp. \textit{(Type it out and explain as you go.)} 
			\item For more details on OpenMDAO, visit the docs, they are very extensive.
			\item \textit{(Celebrate with them about how much they’ve learned, get them excited about what they now can do.)}
			\item \textbf{Assignment: None. It's all done, yay!!!}
		\end{itemize}
	
\end{itemize}

\end{document}
		